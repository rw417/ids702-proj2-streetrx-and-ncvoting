% Options for packages loaded elsewhere
\PassOptionsToPackage{unicode}{hyperref}
\PassOptionsToPackage{hyphens}{url}
%
\documentclass[
]{article}
\usepackage{amsmath,amssymb}
\usepackage{lmodern}
\usepackage{iftex}
\ifPDFTeX
  \usepackage[T1]{fontenc}
  \usepackage[utf8]{inputenc}
  \usepackage{textcomp} % provide euro and other symbols
\else % if luatex or xetex
  \usepackage{unicode-math}
  \defaultfontfeatures{Scale=MatchLowercase}
  \defaultfontfeatures[\rmfamily]{Ligatures=TeX,Scale=1}
\fi
% Use upquote if available, for straight quotes in verbatim environments
\IfFileExists{upquote.sty}{\usepackage{upquote}}{}
\IfFileExists{microtype.sty}{% use microtype if available
  \usepackage[]{microtype}
  \UseMicrotypeSet[protrusion]{basicmath} % disable protrusion for tt fonts
}{}
\makeatletter
\@ifundefined{KOMAClassName}{% if non-KOMA class
  \IfFileExists{parskip.sty}{%
    \usepackage{parskip}
  }{% else
    \setlength{\parindent}{0pt}
    \setlength{\parskip}{6pt plus 2pt minus 1pt}}
}{% if KOMA class
  \KOMAoptions{parskip=half}}
\makeatother
\usepackage{xcolor}
\IfFileExists{xurl.sty}{\usepackage{xurl}}{} % add URL line breaks if available
\IfFileExists{bookmark.sty}{\usepackage{bookmark}}{\usepackage{hyperref}}
\hypersetup{
  pdftitle={Street Price of Lorazepam},
  hidelinks,
  pdfcreator={LaTeX via pandoc}}
\urlstyle{same} % disable monospaced font for URLs
\usepackage[margin=1in]{geometry}
\usepackage{graphicx}
\makeatletter
\def\maxwidth{\ifdim\Gin@nat@width>\linewidth\linewidth\else\Gin@nat@width\fi}
\def\maxheight{\ifdim\Gin@nat@height>\textheight\textheight\else\Gin@nat@height\fi}
\makeatother
% Scale images if necessary, so that they will not overflow the page
% margins by default, and it is still possible to overwrite the defaults
% using explicit options in \includegraphics[width, height, ...]{}
\setkeys{Gin}{width=\maxwidth,height=\maxheight,keepaspectratio}
% Set default figure placement to htbp
\makeatletter
\def\fps@figure{htbp}
\makeatother
\setlength{\emergencystretch}{3em} % prevent overfull lines
\providecommand{\tightlist}{%
  \setlength{\itemsep}{0pt}\setlength{\parskip}{0pt}}
\setcounter{secnumdepth}{-\maxdimen} % remove section numbering
\ifLuaTeX
  \usepackage{selnolig}  % disable illegal ligatures
\fi

\title{Street Price of Lorazepam}
\author{Raza Lamb (Coordinator), Robert Wan (Writer)\\
Team: Erika Fox (Programmer), Minjung Lee (Presenter), Preet Khowaja
(Checker),}
\date{10/24/2021}

\begin{document}
\maketitle

\begin{verbatim}
## [1] "/Users/erikafox/Desktop/StatTeamProject2/team-project-2-streetrx-and-voting-in-nc-purple-team"
\end{verbatim}

\hypertarget{summary}{%
\subsection{Summary}\label{summary}}

Using a hierarchical linear regression model and data reported by
citizens anonymously, we investigated the factors that impact the per mg
price on the street of Lorazepam in the United States. We found that the
street price is most impacted by the dosage strength per unit, with
stronger pills generally costing less per mg. The street prices in most
states are similar, although 6 states have street prices that are
statistically different from the national average.

\hypertarget{introduction}{%
\subsection{Introduction}\label{introduction}}

Some studies place the size of the street market of diverted
pharmaceutical substances at around \$400 billion per year. However, it
is an opaque market where player activities are actively concealed and
pricing dynamics are often not well understood. A good understanding of
the prices of drugs on the street can be helpful for public health
officials as they seek to combat substance abuse issues.

In this study, we investigated on the street price of Lorazepam.
Lorazepam is a benzodiazepine that is used to treat anxiety disorders.
Abuse of Lorazepam has been associated with a number of health issues,
including impaired muscular coordination and memory loss. In the United
States, Lorazepam is a Schedule IV controlled substance that can be
legally acquired only through prescriptions, which creates a demand on
the street for the drug. Specifically, we sought to understand what
factors are associated with the street price of Lorazepam in the United
States, and how street price varies by state.

\hypertarget{data}{%
\subsection{Data}\label{data}}

\hypertarget{data-source}{%
\subsubsection{Data Source}\label{data-source}}

For this analysis, we got our data from StreetRx (streetrx.com).
StreetRx is a web-based citizen reporting tool that allows people to
anonymously report the street price of diverted pharmaceutical
substances they are aware of or have purchased. Using user-generated
data allows us to get insights about an otherwise opaque market.
However, we have to be weary about potentially incorrect values in the
data set because the data points have not been double-checked.

\hypertarget{code-book}{%
\subsubsection{Code Book}\label{code-book}}

\begin{itemize}
\tightlist
\item
  \(ppm\): price per mg or Lorazepam, in USD
\item
  \(state\): the US state that the user entered
\item
  \(USA\_region\): the region of the US based on the state
\item
  \(source\): the source that the user got the price
\item
  \(form\): the formulation of the drug
\item
  \(mgstr\): dosage strength of the drug purchased, in mg / unit
\item
  \(bulk\_purchase\): 1 if the purchase was of 10+ units, 0 otherwise
\end{itemize}

\hypertarget{data-cleaning}{%
\subsubsection{Data Cleaning}\label{data-cleaning}}

Since the data was reported by individual users instead of collected by
professionals, some data points might be incorrect. We found a few
problems with the data set and took measures to address them, which are
listed below.

\begin{enumerate}
\def\labelenumi{\arabic{enumi}.}
\tightlist
\item
  \textbf{Dropped observations with \(ppm\) above \$100/mg}: around 20
  observations have prices of over \$100/mg, with a few as high as
  \$400/mg. The average price of Lorazepam is around \$1/mg in
  pharmacies, so prices in the hundreds are likely due to mistakes in
  reporting. However, since there is not an authoritative source of the
  real price of Lorazepam, we wanted to be conservative in deciding
  which prices were errorneous. By dropping observations with \(ppm\)
  above \$100/mg, we dropped 0.3\% of observations, which we believed
  strike a good balance between keeping the data clean and retaining
  enough information.
\item
  \textbf{Dropped observations with strength above 2.5mg/unit}: two
  observations had strength (dosage strength per unit) above 2.5mg/unit.
  Lorazepam is offered in strengths from 0.5mg/unit to 2.5mg/unit. Since
  only two observations had strength above 2.5mg/unit, we were
  comfortable with removing both observations.
\item
  \textbf{Dropped observations where the \(source\) have fewer than 5
  reported prices of Lorazepam}: because users were allowed to enter
  their source in free text form, some sources were web urls or had
  unique names. This resulted in many distinct values of the \(source\)
  variables with five or fewer observations. A total of 28 observations
  were like this. These observations would not be useful for building
  the model, and we dropped them because they made up less than 0.5\% of
  the data set.
\item
  \textbf{Transformed \(mgstr\) into a factor variable}: because there
  are only 4 distinct values of \(mgstr\), we transformed it into a
  factor variable, named \(mgstr_factor\).
\item
  \textbf{Combined states / territories with 10 or fewer observations
  into one group}: American Samoa, Washington DC, Virgin Islands, and
  Wyoming had 10 or fewer observations. We combined these four states /
  territories into one group to avoid problems caused by small sample
  sizes.
\end{enumerate}

\hypertarget{exploratory-data-analysis}{%
\subsubsection{Exploratory Data
Analysis}\label{exploratory-data-analysis}}

The variable we would like to use as the response variable of the model,
\(ppm\), is a continuous variable, so it would make most sense to use a
linear regression. As such, we first checked if \(ppm\) follows a normal
distribution. It turns out that having removed observations with \(ppm\)
\textgreater{} \$100, the distribution of \(ppm\) was still not normal.
On the other hand, the natural log of \(ppm\), named \(ppm_log\),
followed a normal distribution, as shown in Figure 1. We decided to use
it as the response variable instead.

\begin{center}\includegraphics[width=0.7\linewidth]{lorazepam_price_files/figure-latex/log ppm-1} \end{center}

Having finalized the response variable to use, we moved on to
investigate potential predictor variables for the model. Our data set
includes two location variables, \(state\) and \(USA_region\). This
prompted us to consider creating a hierarchical model group by
geography. We explored weather the average \(ppm_log\) differed by those
two variables, We found that \(ppm_log\) differed substantially by
\(state\), but there were no discernible differences between each
\(USA_region\). As such, we believed it would be helpful to group by
\(state\) in the model but not as helpful to group by \(USA_region\). We
would take another look at the hierarchies when we construct the model.

We then looked at other variables in the data set, namely \(source\),
\(form\), \(mgstr_factor\), and \(bulk_purchase\). All Lorazepams in the
data set come in the form of pill/tablet, so we discarded this variable
for modeling. The remaining three variables all seemed be correlated
with \(ppm_log\) to a degree, so we decided to include all of them in
the null model we would build.

In terms of interaction variables, we did not find many interactions to
be interesting - for most interactions, the pattern of average
\(ppm_log\) did not change between the groups created by the
interactions. However, we found one interaction that could be worth to
further investigate. As shown in Figure 2, the effect of
\(bulk_purchcase\) appeared to differ by state. Lorazepam bought through
bulk purchases had high prices in states like Idaho, whereas it cost
less in states such as Maryland. This suggests that we could explore
wether adding a varying slope by \(bulk_purchase\) would improve the
model.

\begin{center}\includegraphics[width=0.7\linewidth]{lorazepam_price_files/figure-latex/interactions-1} \end{center}

\hypertarget{model}{%
\subsection{Model}\label{model}}

The final model we selected was a hierarchical linear regression model
with varying intercepts for each state. The fixed effects predictors
were source, bulk purchase, and dosage strength as a factor variable,
and there was no interaction variable. In this model, each state has its
own intercept, which is an effect in addition to the common intercept.
The formal representation of the model is: \[
\begin{aligned}
y_{ij} & = (\beta_{0} + \gamma_{0j}) + \beta_1 source_{ij} + \beta_2 bulk\_purchase_{ij} + \beta_3 mgstr\_factor_{ij} + \epsilon_{ij}; \ \ \ i = 1, \ldots, n_j; \ \ \ j = 1, \ldots, 51 \\
\epsilon_{ij} & \sim N(0, \sigma^2) \\
\gamma_{0j} & \sim N(0, \tau_{state}^2)
\end{aligned}
\]

\hypertarget{variable-selection}{%
\subsubsection{Variable Selection}\label{variable-selection}}

The final model shown above was actually the base model we started model
selection with. Since we were sure that we would like to build a
hierarchical model grouped at the state level, we started model
selection with this model, which has no interactions and one varying
intercept by the state variable.

Even though we did not find any interesting interaction for fixed
effects in the EDA, we added interactions terms to the model in order to
ensure that we did not miss important predictors. We performed ANOVA
tests to compare the base model and models with interaction terms. It
turned out that no interaction term we attempted to add improved the
model fit in a statistically significant way. As such, we decided to not
include any interaction term for fixed effected in the final model.

In the EDA, we found that the interaction between \(bulk_purchase\) and
\(state\) could be worth further investigations. As such, we added a
varying slope for \(bulk_purchase\) grouped by \(state\). However, we
got singular fits when we tried to train the model regardless of the
optimizer we used in the lmer() function. This indicated that the model
likely suffered from overfit issues when we added the varying slope. The
structure of the random effects we tried to model was too complex to be
supported by the amount of data we had. That was entirely reasonable
since we had a little over 6,000 observations in the data set and
\(state\) already split the data into 51 groups. Therefore, we resorted
to using only a varying intercept for \(state\) and no varying slope.

\hypertarget{model-asesssment}{%
\subsubsection{Model Asesssment}\label{model-asesssment}}

To assess the four assumptions of linear regression, we first looked at
a scatterplot of residual vs fitted values. Residuals were evenly
scattered around 0 and did not have any pattern with respect to fitted
values. As such, the model passed the independence assumption and the
equal variance assumption. We also did not notice any influential
outliers from the plot. The linearity assumption does not apply to our
model because all predictors are factor variables.

Things become trickier regarding the normality assumption. The x and y
axes of the QQ plot below represent the theoretical and actual z-scores
of each residual. Residuals with theoretical z-scores between -1.5 and
1.5 lie on the QQ line, which translates to roughly 88\% of residuals
following a normal distribution. Residuals with more extreme values tend
to spread out more than what we would expect from a normal distribution.
Since the data set still includes some extreme values for ppm, this is
not unexpected - however, this would be worth diving deeper in future
research and could possibly be resolved as we gather more data and
become more confident on a reasonable range for the street price for
Lorazepam.

\begin{center}\includegraphics[width=0.6\linewidth]{lorazepam_price_files/figure-latex/assess_2-1} \end{center}

\hypertarget{results-and-interpretation}{%
\subsubsection{Results and
Interpretation}\label{results-and-interpretation}}

A table of model coefficients and a plot showing the random intercept of
each state are included in the appendix. For fixed effects, the factor
variable for dosage strength, \(mgstr\_factor\), is the most influential
predictor. Pills with higher dosages generally have a lower price per
milligram than pills with lower dosages. Holding all else equal,
compared to pills with a dosage strength of 0.5mg / unit, 1mg / unit
pills cost 45.05\% less per mg, and 2mg / unit pills cost 64.56\% less
per mg The highest dosage pills, with a strength of 2.5mg / unit,
constitute only 1.7\% of all observations, which suggests that there
might be less supply of pills with such a high dosage. It also means
that we would not be able to infer the ppm of these pills as accurately.
As such, the ppm of 2.5mg / unit pills is slightly higher than the ppm
of 2mg / unit pills - compared to 0.5mg / unit pills, 2.5mg / unit pills
cost only 60.36\% less per mg. At the 0.05 level, other fixed effects
are not statistically significant. However, the coefficient of
\(bulk\_purchase\) is significant at the 0.1 level, and drugs purchased
in bulk are 5.78\% cheaper per mg, holding all else equal.

For random effects, the standard deviation of the varying intercept is
0.1159 while the standard deviation of the residuals is 0.9264. The
varying intercept by state explains 12.51\% of the variation of ppm,
while the fixed effects explain the remaining variations in ppm within
each state. The confidence intervals of the intercept for most states
include 0, so the baseline price in most states are not significantly
different from the average baseline price of the country. However, a few
states have prices that are statistically different from the average
price of the country - all else held equal, baseline prices in NJ, FL,
and CA are much higher than the national average and baseline prices in
MI, PA, and IA are much lower.

\hypertarget{conclusion}{%
\subsection{Conclusion}\label{conclusion}}

In this study, we found that the most influential factor of the street
per mg price of Lorazepam is the dosage strength. For pills with
strengths between 0.5mg / unit and 2mg / unit, higher strength
translates to lower per mg prices. However, pills with the highest
dosage strength, 2.5mg / unit, cost slightly more than pills with 2mg /
unit, which is likely due to limited supply on the street. Purchasing
Lorazepam in bulk does tend to lower the per mg price, yet the effect is
not significant at the 0.05 level, and the magnitude of the effect does
not appear to be scientifically meaningful. The location of the
transaction does not play a meaningful role in determining the street
price of Lorazepam. While a few states have statistically different
baseline per mg prices than the national average baseline, most states
have similar prices.

\hypertarget{limitations}{%
\subsubsection{Limitations}\label{limitations}}

As previously mentioned, one of the biggest limitations of this analysis
is the source of our data, in that all the information we have is
user-generated and therefore unreliable. User-generated data tends to be
full of missing values and bias. Due to the nature of our data, we did
not use a varying slope for state.

\newpage

\hypertarget{appendix-1-model-coefficients}{%
\subsection{Appendix 1: Model
Coefficients}\label{appendix-1-model-coefficients}}

~

ppm log

Predictors

Estimates

CI

p

(Intercept)

1.63

1.57~--~1.69

\textless0.001

source {[}Heard it{]}

-0.06

-0.13~--~0.01

0.122

source {[}Internet{]}

-0.08

-0.22~--~0.05

0.232

source {[}InternetPharmacy{]}

0.15

-0.02~--~0.33

0.083

source {[}Personal{]}

0.00

-0.05~--~0.06

0.879

bulk purchase {[}1 Bulkpurchase{]}

-0.06

-0.12~--~0.00

0.066

mgstr factor {[}1{]}

-0.60

-0.65~--~-0.55

\textless0.001

mgstr factor {[}2{]}

-1.04

-1.11~--~-0.97

\textless0.001

mgstr factor {[}2 5{]}

-0.93

-1.11~--~-0.74

\textless0.001

Random Effects

σ2

0.86

τ00 state

0.01

ICC

0.02

N state

51

Observations

6075

Marginal R2 / Conditional R2

0.140 / 0.153

\newpage

\hypertarget{appendix-2-varying-intercept-of-each-state}{%
\subsection{Appendix 2: Varying Intercept of Each
State}\label{appendix-2-varying-intercept-of-each-state}}

\begin{center}\includegraphics{lorazepam_price_files/figure-latex/unnamed-chunk-3-1} \end{center}

\newpage

\hypertarget{appendix-3-residuals-vs-fitted-values-of-final-model}{%
\subsection{Appendix 3: Residuals vs Fitted Values of Final
Model}\label{appendix-3-residuals-vs-fitted-values-of-final-model}}

\begin{center}\includegraphics[width=1\linewidth]{lorazepam_price_files/figure-latex/unnamed-chunk-4-1} \end{center}

\end{document}
